%%%%%%%%%%%%%%%%%%%%%%%%%%%%%%%%%%%%%%%%%%%%%%%%%%%%%%%%%%%%%%%%%%%%%%%%
%                                                                      %
%     File: Thesis_Glossary.tex                                        %
%     Tex Master: Thesis.tex                                           %
%                                                                      %
%     Author: Andre C. Marta                                           %
%     Last modified : 29 Jun 2022                                      %
%                                                                      %
%%%%%%%%%%%%%%%%%%%%%%%%%%%%%%%%%%%%%%%%%%%%%%%%%%%%%%%%%%%%%%%%%%%%%%%%
%
% The definitions can be placed anywhere in the document body
% and their order is sorted by <symbol> automatically when
% calling makeindex in the makefile
%
% The \glossary command has the following syntax:
%
% \glossary{entry}
%
% The \nomenclature command has the following syntax:
%
% \nomenclature[<prefix>]{<symbol>}{<description>}
%
% where <prefix> is used for fine tuning the sort order,
% <symbol> is the symbol to be described, and <description> is
% the actual description.

% ----------------------------------------------------------------------

\glossary{name={\textbf{MDO}},description={Multi-Disciplinar Optimization}}

\glossary{name={\textbf{CFD}},description={Computational Fluid Dynamics}}

\glossary{name={\textbf{CSM}},description={Computational Structural Mechanics}}

