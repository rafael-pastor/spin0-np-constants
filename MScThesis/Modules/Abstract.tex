%%%%%%%%%%%%%%%%%%%%%%%%%%%%%%%%%%%%%%%%%%%%%%%%%%%%%%%%%%%%%%%%%%%%%%%%
%                                                                      %
%     File: Thesis_Abstract.tex                                        %
%     Tex Master: Thesis.tex                                           %
%                                                                      %
%     Author: Andre C. Marta                                           %
%     Last modified :  2 Jul 2015                                      %
%                                                                      %
%%%%%%%%%%%%%%%%%%%%%%%%%%%%%%%%%%%%%%%%%%%%%%%%%%%%%%%%%%%%%%%%%%%%%%%%

\section*{Abstract}

% Add entry in the table of contents as section
\addcontentsline{toc}{section}{Abstract}

This thesis explores the computation of NP constants for a spin-0 field in Minkowski spacetime, focusing on their behavior near spatial and null infinity. To achieve this, Friedrich's $i^0$-cylinder framework is utilized. Under the assumption that the initial data meets specific regularity conditions, allowing for the analytical extension of the field to critical sets, the study finds that the NP constants at future null infinity $\mathscr{I}^{+}$ and past null infinity $\mathscr{I}^{-}$ remain independent of each other. In other words, the classical NP constants at $\mathscr{I}^{\pm}$ are determined by different segments of the initial data, which are defined on a Cauchy hypersurface.
In contrast, by introducing a slight generalization known as the $i^0$-cylinder NP constants, the need for the regularity condition is eliminated. These modified NP constants yield conserved quantities at $\mathscr{I}^{\pm}$ solely determined by a specific part of the initial data, which, in turn, corresponds to the terms governing the regularity of the field. Moreover, the conservation laws linked to the NP constants are exploited to construct heuristic asymptotic-system expansions in flat space, effectively capturing the impact of logarithmic terms at the critical sets. This characteristic proves to be fascinating in the study of evolution equations using the $i^0$-cylinder framework.

\vfill

\textbf{\Large Keywords:} Newman-Penrose constants, null infinity, Friedrich Cylinder, asymptotic system, spin-0 field.

