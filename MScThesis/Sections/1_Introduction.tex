%%%%%%%%%%%%%%%%%%%%%%%%%%%%%%%%%%%%%%%%%%%%%%%%%%%%%%%%%%%%%%%%%%%%%%%%
%                                                                      %
%     File: Thesis_Introduction.tex                                    %
%     Tex Master: Thesis.tex                                           %
%                                                                      %
%     Author: Andre C. Marta                                           %
%     Last modified : 13 May 2019                                      %
%                                                                      %
%%%%%%%%%%%%%%%%%%%%%%%%%%%%%%%%%%%%%%%%%%%%%%%%%%%%%%%%%%%%%%%%%%%%%%%%

\chapter{Introduction}
\label{chapter:introduction}

%%%%%%%%%%%%%%%%%%%%%%%%%%%%%%%%%%%%%%%%%%%%%%%%%%%%%%%%%%%%%%%%%%%%%%%%
\section{Motivation}
\label{section:motivation}

General relativity, developed by Albert Einstein, is a comprehensive theory of gravitation that fundamentally alters our understanding of the force of gravity. It portrays gravity not as a Newtonian force, but rather as a consequence of the curvature of spacetime caused by the presence of mass and energy. According to this theory, objects move through spacetime along paths dictated by this curvature, giving rise to the illusion of gravitational force.
\\
One intriguing prediction of general relativity is the existence of black holes, celestial objects with such immense gravitational pull that nothing, not even light, can escape their grasp. In the context of general relativity, the presence of a black hole causes spacetime to contort and deform, leading to bizarre phenomena like time dilation and the bending of light rays. To an observer located at a significant distance from a black hole, its appearance is analogous to that of a classical particle. This resemblance allows us to characterize a black hole by three primary quantities: its total mass, electrical charge, and spin. Remarkably, black holes that possess identical properties are indistinguishable, as no measurements can be made to discern their uniqueness \cite{HawMalStr16} — a concept known as the "No Hair" theorem.
\\
Understanding how different objects interact with black holes requires an exploration of the conservation laws that govern their behavior. By observing the initial and final state of an object that falls into a black hole, we can deduce a few of its properties through the lens of conservation. However, the reductionist nature of black holes poses a challenge—the wealth of information carried by stars and planets of various shapes and sizes is reduced to a mere three numbers. Consequently, a significant amount of information is lost, leading to a perplexing conundrum known as the information paradox \cite{HawMalStr16}.
\\
Expanding upon the topic, it is crucial to consider binary systems as significant sources of gravitational waves. Gravitational waves are ripples in the fabric of spacetime that propagate outward, carrying energy away from their source. Binary systems, composed of two massive objects orbiting around each other, emit gravitational waves as a result of their orbital motion. These waves can be detected and studied, providing valuable insights into the dynamics of spacetime and further confirming the predictions of general relativity.
\\
In the study of dynamical spacetime, researchers investigate the behavior of spacetime itself when subjected to the presence of matter and energy. The dynamics of spacetime can be studied by employing mathematical frameworks such as the theory of general relativity. By understanding the intricate interplay between matter, energy, and spacetime curvature, scientists gain a deeper understanding of how the fabric of the universe evolves and changes in response to different physical phenomena \cite{DaiVal02}.
\\
In summary, general relativity revolutionizes our comprehension of gravity by describing it as a consequence of spacetime curvature caused by mass and energy. Black holes, characterized by their immense gravitational pull, serve as fascinating objects that challenge our understanding of information conservation. The information paradox arises from the reduction of complex objects to a mere three numbers, leading to the potential loss of vast amounts of information. However, recent concepts like soft "hair" propose avenues for exploring the consumption history of black holes and potentially resolving the information paradox. Additionally, binary systems play a crucial role in generating gravitational waves, allowing us to probe the dynamics of spacetime and deepen our understanding of the universe's fundamental nature. In this thesis we will focus on the calculation of Newman-Penrose constants on a framework where we will focus on two types of infinity: null infinity denoted by $\mathscr{I}$ and spatial infinity denoted by the symbol $i^0$.


%%%%%%%%%%%%%%%%%%%%%%%%%%%%%%%%%%%%%%%%%%%%%%%%%%%%%%%%%%%%%%%%%%%%%%%%
\section{Global structure of spacetimes}
\label{section:Global structure of spacetimes}

Roger Penrose brought to the field of general relativity the notion of conformal transformation, which made a significant impact in the geometric understanding of infinity. This was crucial for the development of theory of asymptotics, which arises a question of whether a smooth conformal extension which attaches a boundary - conformal boundary represents points at infinity - to the spacetime is shared by a larger class of spacetimes. This question leads to the notion of \textit{asymptotic simplicity} \cite{Val16}. In the context of asymptotic simplicity, spatial infinity (denoted as $i^0$) represents the region at an infinite distance from the central object or system under consideration. It provides a framework to analyze the properties of spacetime far away from the gravitational source. Similarly, null infinity (denoted as $\mathscr{I}$) represents the region in the future or past. It corresponds to the points reached by light rays that have traveled an infinite distance from their source. This simplicity allows for the application of mathematical techniques and tools to study the behavior of physical fields, gravitational waves, and the conservation laws in these simplified regimes.
Asymptotic simplicity is a concept in general relativity that refers to the behavior of spacetime at infinity. It characterizes the way spacetime and its geometry approach a simple and well-defined structure as we move to spatial infinity or null infinity.
\\
The geometric understanding of infinity also contributed to the development of gravitational radiation, taking a step forward in the mathematical understanding of gravitational waves. Although, customary in numerical approaches to general relativity, wave forms are computed at large radius, from first principles point of view they should be computed at null infinity $\mathscr{I}$. To do so, the Einstein field equations need to be expressed in terms of suitably rescaled fields, so one can evaluate the fields at $\mathscr{I}$. Technically this is done by a conformal transformation. In general relativity, conformal transformations are used to describe the behavior of physical systems under changes in the scale of spacetime. These transformations preserve the local structure of spacetime and its overall shape (angles between vectors). The original metric, which we refer to as the physical metric, is denoted by $\tilde{g}$. We consider a transformation to an unphysical metric, $g$, which is given by 
\begin{equation}\label{eq:conftrans}
	g_{ab} = \Xi^2 \tilde{g}_{ab},
\end{equation}
$\Xi$ is a smooth function that approaches zero as the distance from the source increases. This transformation, given by \eqref{eq:conftrans}, preserves angles, making it appropriate to describe it as conformal. H. Friedrich introduced the \textit{conformal Einstein field
equations} (CEFE), a formulation designed that is in accordance with
the approach of R. Penrose.
\medskip

A prototypical example is the conformal extension of the Minkowski spacetime which will be discussed in the following.  One starts with the Minkowski metric line-element,
\begin{equation}\label{eq:metricMink-tr}
	d \tilde{s}^2=-d \tilde{t}^2+d \tilde{r}^2+\tilde{r}^2 d
        \Omega^2,
\end{equation}
where $(\tilde{t}, \tilde{r})$ $\in$ $(-\infty,+\infty) \times[0,+\infty)$ and $d \Omega^2$ represents the standard metric on $\mathbb{S}^2$. To get a conformal extension we need to do a coordinate transformation, corresponding to the advanced and retarded  times, $\tilde{u}=\tilde{t}-\tilde{r}$ $\&$ $\tilde{v}=\tilde{t}+\tilde{r}$,  substituting equation \eqref{eq:metricMink-tr}.
\begin{equation}\label{eq:metricMink-tr1}
	d \tilde{s}^2=-d \tilde{u} d \tilde{v}+\frac{(\tilde{u}-\tilde{v})^2}{4} d \Omega^2.
\end{equation}
For the compactification, we need to introduce the following: $u = \tan U$ $\&$ $v = \tan V$, where $U$, $V$ $\in$ $(- \pi/2, \pi/2)$. Now, we are able to identify the conformal metric,
$ds$. Using \eqref{eq:metricMink-tr}, we obtain
$$d s^2=-4 d U d V+\sin ^2(V-U) d \Omega^2$$ where
$$d s^2=\Xi^2 d \tilde{s}^2$$
with $\Xi=2 \cos U \cos V$. Given the
domain of $U$ and $V$, we introduce the following, $T=V+U$ $\&$ $\psi=V-U$. The domain of $(T, \psi)$ is $(-\pi, \pi)$, with
\begin{equation}\label{eq:metricMink-cf}
	d s^2=-d T^2+d \psi^2+\sin ^2 \psi d \Omega^2,
\end{equation}
which is the metric of the Einstein static universe. The conformal boundary is given by $\psi = \pi/2$, which is the cylinder $\mathbb{R} \times \mathbb{S}^2$, the Einstein Cylinder. The purpose of this thesis is  to study what happens at infinity, and in order to do that we will focus on the region where $\Xi = 0$. This condition gives us the following regions, which are presented in the following table
\begin{center}
    \begin{tabular}{ |c|c|c| }
      \hline
      Region & Name & Symbol \\
     \hline
     $\tilde{r} \rightarrow \infty \enspace \;  \text{with} \; \;\;\;\;
     |\tilde{t}| $<$ \infty$ & Spatial Infinity & $i^{0}$ \\
     \hline 
     $\tilde{t} \rightarrow \pm \infty$  \;\; with \;\; $\tilde{r}$<$\infty$
     \enspace & Future/Past Timelike Infinity & $i^{\pm}$ \\
     \hline
     $\tilde{r} \rightarrow \infty, \enspace \tilde{t} \rightarrow \infty \enspace
     \text{with} \;\; |u|$<$\infty$ &  Future Null-infinity & $\mathscr{I}^+$ \\ 
     \hline
     $\tilde{r} \rightarrow\infty, \enspace \tilde{t} \rightarrow-\infty \;\;
     \text{with} \enspace |v|$<$\infty$ & Past Null-infinity & $\mathscr{I}^-$ \\
     \hline
    \end{tabular}
    \end{center}
The visual representation of this is a Penrose Diagram, depicted in Fig.1
\begin{figure}[h]
	\centering \includegraphics[width =0.3\textwidth]{Penrose diagram.pdf}
    \caption{Penrose Diagram - Representation of the standard
      compactification of the Minkowski spacetime alongside the curves of constant
      time, solid black lines, and the curves of constant r, dotted
      black lines.}
\end{figure}
\\
Additionally, H. Friedrich proposed another conformal representation of Minkowski spacetime specifically adapted for spatial infinity, which will be one used in this work. By applying the following change of coordinates $\tilde{t} = \frac{\tau}{\rho(1-\tau^{2})}$, \enspace $\tilde{\rho} = \frac{1}{\rho(1-\tau^{2})}$ we arrive at this representation. Therefore,
$$ \gamma=-d \tau^2+\frac{\left(1-\tau^2\right)}{\rho^2} d\rho^2-\frac{\tau}{\rho}(d \rho d \tau+d \tau d \rho)+d \Omega^2 $$
We are now in the position to say that the conformal factor $\Theta$ is given by,
\begin{equation}\label{eq:gamma}
	\gamma = \Theta^2 \tilde{\eta}.
\end{equation}
\begin{equation}\label{eq:conf-factor}
	\Theta = \rho(1-\tau^2).
\end{equation}
The range of the $\tau$ coordinate is,  
$$ -1 \le \tau \le 1 $$ 
with $\rho$ > 0. Therefore, light travels towards infinity to the places where $\tau = \pm 1$ in the conformal extension. So, in this representation,
$$ \mathscr{I}^+ \equiv \{ \tau = 1 \}, \enspace \mathscr{I}^- \equiv \{ \tau = -1 \}.$$
The sets where future and past null-infinities touch spatial infinity are named the critical sets and are given by,
$$ {I}^+ = \{ \tau = 1, \enspace \rho = 0\}, \enspace {I}^- = \{ \tau = -1, \enspace \rho = 0\}$$

%%%%%%%%%%%%%%%%%%%%%%%%%%%%%%%%%%%%%%%%%%%%%%%%%%%%%%%%%%%%%%%%%%%%%%%%
\section{Newman-Penrose constants}
\label{section:Newman-Penrose Constants}
The Newman-Penrose constants, originally introduced in \cite{NewPen68}, are quantities defined on null-infinity that obey conservation laws for asymptotically flat gravitational fields. For linear fields propagating in flat spacetime, there exists an infinite number of conservation laws. For example, in ordinary Electromagnetic (EM) theory with a spin-1 field, the total charge is conserved. In the linearized gravitational theory, which involves a spin-2 field, the total mass, linear momentum, and angular momentum are also conserved. In our case, we are interested in studying spin-0 fields, which correspond to solutions of the wave equation.
\\
The NP constants form an infinite hierarchy of conserved quantities for linear equations, including the spin-1, spin-2, and spin-0 fields. Newman and Penrose demonstrated that these constants can be expressed as the product of the square of the dipole moment and the difference between the monopole and the quadrupole moments, as shown in \cite{DaiVal02}. However, in the full non-linear gravitational theory, the conservation of mass and momentum no longer holds.
\\
One intriguing question is whether the NP constants are zero for stationary spacetimes. Remarkably, for the Kerr solution and the Schwarzschild spacetime, the NP constants do vanish \cite{Bac10}, \cite{BaiZhoGonXueXiaoLau07}. The magnitude of these constants provides insights into the residual radiation present in the spacetime following a black hole collision \cite{DaiVal02}. As the NP constants retain their values along null-infinity, they offer valuable information about the behavior of black hole collisions at later times.
\\
Turning our attention to the interpretation of these charges, the NP constants are considered a set of conserved charges at null infinity \cite{NewPen68}. These charges are computed as 2-surface integrals at cuts $\mathcal{C} \approx \mathbb{S}^2$ of null infinity $\mathscr{I}$. In the linear theory, an infinite hierarchy of these conserved quantities exists, while in the non-linear theory of General Relativity, only ten quantities remain conserved \cite{NewPen68}.
\\
The interpretation of these charges is still a subject of debate [\cite{PenRin84}, \cite{DaiVal02}, \cite{Bac10}]. However, their conservation holds in general asymptotically flat spacetimes, even when the dynamics involve complex phenomena like black hole collisions \cite{DaiVal02}. Recent interest in asymptotic quantities, particularly the Bondi-Metzner-Sachs (BMS) charges, has emerged due to their connection with the concept of black hole soft hair [\cite{HawMalStr16}, \cite{HawMalStr17}, \cite{HeLyMi15}]. Understanding the relationship between conserved quantities at past and future null infinity is a crucial aspect of these discussions. However, the resolution of the singular nature of spatial infinity poses challenges in matching the conserved quantities.
\\
Therefore, the study of the Newman-Penrose constants and their conservation offers valuable insights into the dynamics of gravitational fields, particularly in black hole collisions and asymptotically flat spacetimes. These constants provide information about the residual radiation and behavior of the system at later times, shedding light on the intricate nature of spacetime and the conservation laws that govern it.
\\
The peeling theorem, one of the most emblematic results in the classical theory of asymptotics in general relativity, has played a very important role in the development of the modern notion of gravitational radiation. It is usually formulated within the context of asymptotically simple spacetimes, as defined in Section 10.2 of \cite{Val16}.
\\
An inspection of the proof of the peeling theorem reveals that, in fact, it is only necessary to assume that the conformal extension is $C^{4}$.  In view of this, the question of the existence and genericity of spacetimes satisfying the peeling behavior can be rephrased in terms of the construction of asymptotically flat spacetimes with, at least, this minimum of differentiability \cite{GasVal17}. Penrose's compactification procedure, when applied to the Minkowski spacetime, yields a fully smooth conformal extension. However, for spacetimes with nonvanishing mass, such as the Schwarzschild spacetime, the conformal structure degenerates at spatial infinity. This degeneration occurs because spatial infinity can be viewed as the final point of the generators of null infinity, either in the past or future direction. As a result, it is natural to expect that the behavior of the gravitational fields near spatial infinity will somehow reflect the peeling properties of the spacetime \cite{Pen65a}.
\\
The peeling theorem, documented in [\cite{Sac61}, \cite{BonBurMet62}, \cite{NewPen62}], represents a significant outcome in the study of asymptotic behavior in general relativity. It characterizes the decay of the Weyl tensor in the asymptotic region of the spacetime, signifying the gradual "peeling" of gravitational radiation. This theorem deepens our understanding of the intricate nature of gravitational radiation and its properties in the far-reaching regions of the spacetime.
\\
Hence, the peeling theorem and the Newman-Penrose constants are interconnected concepts that deepen our understanding of the gravitational radiation and its properties in asymptotically flat spacetimes.
%%%%%%%%%%%%%%%%%%%%%%%%%%%%%%%%%%%%%%%%%%%%%%%%%%%%%%%%%%%%%%%%%%%%%%%%

\chapter{Cylinder at $i^0$}
\label{chapter:cylinder}

In Penrose's conformal approach \cite{Pen63}, the investigation of the gravitational field's decay and the asymptotic structure of spacetime is conducted not with respect to the physical spacetime $(\tilde{\mathcal{M}}, \tilde{\boldsymbol{g}})$ that satisfies the Einstein field equations $\tilde{R}_{a b}=0$. Instead, it is examined in terms of conformally related spacetime $(\mathcal{M}, \boldsymbol{g})$, referred to as the unphysical spacetime, where $\boldsymbol{g}=\Omega^2 \tilde{\boldsymbol{g}}$. The conformal factor $\Omega$ plays a crucial role as a boundary defining function. Specifically, it defines the set of points $\mathscr{I}$ in the unphysical manifold where $\Omega = 0$ while ensuring that $d\Omega \neq 0$.
\\
Within the conformal structure of asymptotically flat spacetimes, a distinguished point is spatial infinity $i^0$, characterized by $\Omega = 0$ and $d\Omega = 0$. Extensive discussions on the conformal approach and the conformal Einstein field equations can be found in [\cite{Val16}, \cite{Fra04}, \cite{Fri02}]. This comprehensive exploration delves into the nuances of the conformal methodology and its implications for understanding the asymptotic behavior and structure of spacetime.
%%%%%%%%%%%%%%%%%%%%%%%%%%%%%%%%%%%%%%%%%%%%%%%%%%%%%%%%%%%%%%%%%%%%%%%%

\section{The $i^0$ - cylinder representation in Minkowski  spacetime}
\label{the $i^0$ cylinder}

Consider the spherical polar coordinates denoted by $(\tilde{t}, \tilde{\rho}, \vartheta^A)$ with $A = 1, 2$, where $\vartheta^A$ represents a set of coordinates on $\mathbb{S}^2$. In this coordinate system, the metric of \textit{physical} Minkowski spacetime is given by $\tilde{\eta}$
\begin{equation}\label{eq:physicalMikmetric}
\tilde{\boldsymbol{\eta}}=-\mathbf{d} \tilde{t} \otimes \mathbf{d} \tilde{t}+\mathbf{d} \tilde{\rho} \otimes \mathbf{d} \tilde{\rho}+\tilde{\rho}^2 \boldsymbol{\sigma}.
\end{equation}
In the specific range, $\tilde{t} \in (-\infty, \infty)$, $\tilde{\rho} \in [0, \infty)$, where $\sigma$ denotes the standard metric on $\mathbb{S}^2$, we introduce \textit{unphysical} spherical polar coordinates $(t, \rho, \vartheta^A)$ as an intermediate step towards obtaining the desired conformal representation
\begin{equation}\label{eq:unphysicalCoords}
t=\frac{\tilde{t}}{\tilde{\rho}^2-\tilde{t}^2}, \quad \rho=\frac{\tilde{\rho}}{\tilde{\rho}^2-\tilde{t}^2}.
\end{equation}
By expressing the physical Minkowski metric $\tilde{\eta}$ in terms of the unphysical spherical polar coordinates $(t, \rho, \vartheta^A)$, one can easily recognize the \textit{inversion} conformal representation of the Minkowski spacetime $(\mathbb{R}^4, \eta)$, where:
\begin{equation}\label{eq:metricRelation}
  \boldsymbol{\eta} = \Xi^2 \boldsymbol{\tilde{\eta}},
\end{equation}
with
\begin{equation}\label{eq:Unphysicalspacetimemetric}
  \boldsymbol{\eta}=-\mathbf{d} t \otimes \mathbf{d} t+\mathbf{d} \rho \otimes \mathbf{d} \rho+\rho^2 \boldsymbol{\sigma}, \quad \Xi=\rho^2-t^2
\end{equation}
In this conformal representation, where $t \in (-\infty, \infty)$, $\rho \in [0, \infty)$, the spatial infinity and the origin undergo an interchange. This implies that $i^0$ is represented by the point $(t = 0, \rho = 0)$ in $(\mathbb{R}^4, \eta)$. Introducing the coordinates $(\tau, \rho, \vartheta^A)$, where $t = \rho \tau$, and considering the conformal metric $\boldsymbol{g} = \rho^{-2} \boldsymbol{\eta}$, we obtain:
\begin{equation}\label{eq:unphysicalmetricMink}
\boldsymbol{g}=-\mathbf{d} \tau \otimes \mathbf{d} \tau+\frac{\left(1-\tau^2\right)}{\rho^2} \mathbf{d} \rho \otimes \mathbf{d} \rho-\frac{\tau}{\rho} \mathbf{d} \rho \otimes \mathbf{d} \tau-\frac{\tau}{\rho} \mathbf{d} \tau \otimes \mathbf{d} \rho+\boldsymbol{\sigma}
\end{equation}
In this context the \textit{unphyical metric $\boldsymbol{g}$} is related to the physical metric through the following relationship:
\begin{equation}\label{eq:gandtheta}
\boldsymbol{g}=\Theta^2 \tilde{\boldsymbol{\eta}}, \quad \text { where } \quad \Theta:=\frac{\Xi}{\rho}=\rho\left(1-\tau^2\right).
\end{equation}
The unphysical metric $\boldsymbol{g}$, also known as the $i^0$ - cylinder metric, is associated with the coordinates $(\tau, \rho, \vartheta^A)$, which are commonly referred to as the $F$ - coordinates system.
The F-coordinate system can be related to the physical polar coordinates through the following relation:
\begin{align}\label{Ftophys}
\tau = \frac{\tilde{t}}{\tilde{\rho}}, \qquad \rho = \frac{\tilde{\rho}}{\tilde{\rho}^2-\tilde{t}^2},
\end{align}
the inverse transformation is given by
\begin{align}\label{phytoF}
\tilde{t} = \frac{\tau}{\rho (1-\tau^2)}, \qquad \tilde{\rho}=\frac{1}{\rho (1-\tau^2)}.
\end{align}
Upon unwrapping the definitions, the conformal factor $\Theta$ in $F$-coordinatesystem and physical coordinates can be expressed as follows:
\begin{align}
\Theta = \rho (1-\tau^2) = \frac{1}{\tilde{\rho}}
\end{align}
The inverse transformation in \eqref{phytoF} can be written as
\begin{align}
\tilde{t}=\frac{\tau}{\Theta}, \qquad \tilde{\rho}= \frac{1}{\Theta}.
\end{align}
Furthermore, it is worth noting that the physical retarded and advanced times, denoted as $\tilde{u}:=\tilde{t}- \tilde{\rho}$ and 
\\$ \tilde{v}:= \tilde{t}+ \tilde{\rho}$, respectively, can be related to the unphysical advanced and retarded times as follows:
\begin{align}\label{eq:UnphysPhysAdvRet}
v:=t-\rho=-\rho(1-\tau)=\tilde{v}^{-1}, \qquad u:=t+\rho=\rho(1+\tau)=-\tilde{u}^{-1}.
\end{align}
In this conformal representation of the Minkowski spacetime, future and past null infinity can be identified at the following locations:
\begin{align}
\mathscr{I}^{+} \equiv \{ p \in \mathcal{M} \; \rvert\; \tau(p) =1\}, \qquad \mathscr{I}^{-} \equiv \{ p \in \mathcal{M} \; \rvert \;\tau(p) =-1\}.
\end{align}
The term $i^0$ - cylinder derives from the observation that spatial infinity is mapped to an extended set $I \approx \mathbb{R}\times \mathbb{S}^2$.
\begin{align*}
I \equiv \{ p \in \mathcal{M} \; \rvert \;\; |\tau(p)|<1, \;\rho(p)=0\}, \qquad I^{0} \equiv \{ p \in \mathcal{M}\; \rvert \;\tau(p)=0, \; \rho(p)=0\},
\end{align*}
where the regions spatial and null infinity meet were already defined in section 1.2.\\
To facilitate the upcoming discussion, we introduce the following adapted $\boldsymbol{g}$ - null frame:
\begin{equation}\label{eq:Fframe}
\boldsymbol{e}=(1+\tau) \boldsymbol{\partial}_\tau-\rho \boldsymbol{\partial}_\rho, \quad \underline{\boldsymbol{e}}=(1-\tau) \boldsymbol{\partial}_\tau+\rho \boldsymbol{\partial}_\rho, \quad \boldsymbol{e}_{\boldsymbol{A}} \quad \text { with } \quad \boldsymbol{A}=\{\uparrow, \downarrow\},
\end{equation}
here, the $\boldsymbol{e}_{\boldsymbol{A}}$ repesents a complex null frame on $\mathbb{S}^2$, accompanied by the associated coframe $\boldsymbol{\omega}^{\boldsymbol{A}}$, enabling us to express the standard metric on $\mathbb{S}^2$ as follows:
\begin{align}
\bm\sigma=2(\bm\omega^{\uparrow}\otimes \bm\omega^{\downarrow}+\bm\omega^{\downarrow}\otimes \bm\omega^{\uparrow}).
\end{align}
\begin{remark}
  To ensure clarity and avoid confusion, we designate the NP - frames hinged at $\mathscr{I}^{\pm}$ as indicated by the symbols $\pm$. Consequently, the elements of the frame on $\mathbb{S}^2$ are labelled with the symbols $\uparrow \downarrow$.
\end{remark}

To differentiate it from other frames, we refer to the tetrad $\{\boldsymbol{e}, \underline{\boldsymbol{e}}, \boldsymbol{e}_{\boldsymbol{A}}\}$ as the $F$ - frame. Notably, in terms of the $F$ - frame, the unphysical metric $\boldsymbol{g}$ can be represented as follows:
\begin{equation}\label{eq:UnphysicalMetricNullTetrad}
  g_{a b}=e_{(a} \underline{e}_{b)}-\omega_{(a}^{\uparrow} \omega_{b)}^{\downarrow}
\end{equation}
the tetrad normalization condition can be expressed as $e_a \underline{e}^a=-\omega^{\uparrow}_ae_{\downarrow}^a=-2$, ensuring that all other contractions vanish. Similarly, the $\tilde{\boldsymbol{\eta}}$ - null frame, denoted as $\{L, \; \underline{L},\; \tilde{\boldsymbol{e}}_{\boldsymbol{A}}\}$, is defined as follows:
\begin{equation}
  L=\boldsymbol{\partial}_{\tilde{t}}+\boldsymbol{\partial}_{\tilde{\rho}}, \quad \underline{L}=\boldsymbol{\partial}_{\tilde{t}}-\boldsymbol{\partial}_{\tilde{\rho}}, \quad \tilde{\boldsymbol{e}}_{\boldsymbol{A}}=\tilde{\rho}^{-1} \boldsymbol{e}_{\boldsymbol{A}}
\end{equation}
the physical null - frame, consisting of vectors $L$, $\underline{L}$, will be denoted as the physical null - frame. It is important to note that all quantities associated with the physical spacetime will be marked with a tilde over the symbol. For instance, quantities like $\tilde{\phi}$ represent physical fields, while $\phi$ corresponds to unphysical fields that have been conformally rescaled.

\medskip
The NP-frame, another unphysical frame, holds significance in defining the NP constants. The relationship between the F-frame and the NP-frame in Minkowski spacetime was derived and discussed in \cite{GasKro16d}, with further insights available in \cite{ValAli22}. By leveraging the findings in \cite{GasKro16d} along with the expressions presented in this section, we can establish the following proposition, which elucidates the connection between these three frames:
 

\begin{proposition}\label{Prop:NPtoFgauge}
  The NP-frame, which is hinged at $\mathscr{I}^{\pm}$, the F-frame, and the standard physical frame for the Minkowski spacetime are interconnected through the following relationship:
\begin{align*}
  \text{\emph{NP hinged at} $\mathscr{I}^{+}$}:& \quad\boldsymbol{e}^{+} = 4(\Lambda_{+})^{2} \boldsymbol{e} = \Theta^{-2} L, && \underline{\boldsymbol{e}}^{+}=
  \tfrac{1}{4}(\Lambda_{+})^{-2}\underline{\boldsymbol{e}} = \underline{L}, && \boldsymbol{e}_{\boldsymbol{A}}^{+}= \boldsymbol{e}_{\boldsymbol{A}}= \Theta^{-1}\tilde{\boldsymbol{e}}_{\boldsymbol{A}}\\ \text{\emph{NP hinged at} $\mathscr{I}^{-}$}:& \quad\boldsymbol{e}^{-} =
  \tfrac{1}{4}(\Lambda_{-})^{-2} \boldsymbol{e} = L, && \underline{\boldsymbol{e}}^{-}= 4(\Lambda_{-})^{2}\underline{\boldsymbol{e}} = \Theta^{-2} \underline{L}, && \boldsymbol{e}_{\boldsymbol{A}}^{-}= \boldsymbol{e}_{\boldsymbol{A}}= \Theta^{-1}\tilde{\boldsymbol{e}}_{\boldsymbol{A}}.
\end{align*}
The conformal factor $\Theta$ and boost parameter $\kappa$ can be expressed in F-coordinates and physical coordinates as follows:
\begin{align}\label{eq:CF-thetaAndBoostParameter}
  \Theta := \rho (1-\tau^2) = \frac{1}{\tilde{\rho}}, \qquad \varkappa := \frac{1+\tau}{1-\tau} = -\frac{\tilde{v}}{\tilde{u}}.
\end{align}
The Lorentz transformation that connects the NP and F-frames can be expressed in terms of the following quantities:
\begin{align}\label{eq:LorentzTransf}
  (\Lambda_{+})^{2}:= \Theta^{-1}\varkappa^{-1}= \rho^{-1}(1+\tau)^{-2}, && (\Lambda_{-})^{2}:= \Theta^{-1}\varkappa= \rho^{-1}(1-\tau)^{-2}.
\end{align}
\end{proposition}
\begin{remark}
  It should be noted that the NP-frame is not a null tetrad for the physical metric $\tilde{\boldsymbol{\eta}}$, but rather with respect to a conformally related metric $\boldsymbol{g'}=\vartheta^2\tilde{\boldsymbol{\eta}}$, where $\vartheta$ represents a conformal factor. In the case of the Minkowski spacetime, it turns out that this conformally related metric $\boldsymbol{g'}$ coincides with the $i^0$-cylinder representation $\boldsymbol{g}$. Specifically, in general, we have $\boldsymbol{g'} = \kappa^2\boldsymbol{g}$, where  $\kappa=1$ for the specific case of the Minkowski spacetime.
  Formal asymptotic expansions for the conformal transformation $\kappa$ and the Lorentz transformation that relates the NP and F-frames have been computed for time-symmetric initial data in asymptotically flat spacetimes, as described in \cite{FriKan00}.
\end{remark}
%%%%%%%%%%%%%%%%%%%%%%%%%%%%%%%%%%%%%%%%%%%%%%%%%%%%%%%%%%%%%%%
\section{Linear model equations}
\label{sec:LinearModelEquations}
Though the CEFEs have made significant strides in the mathematical study of spacetimes and numerical evolutions, their use in solving physical issues has been surprisingly sparse. This is clear from the fact that there is currently no literature on the use of linear perturbation theory to study the CEFEs (and perhaps as a result of this). To characterize gravitational radiation using conventional (linear) metrics, such an analysis is required.\\
The Bianchi identities, which offer a collection of evolution and constraint equations for the Weyl curvature, form the basis of the CEFEs. The spin-2 equation in a fixed backdrop spacetime, which is analogous to linearizing the Weyl sector of the CEFEs in spinorial form, is one approach to the linearized issue. The linearized metric formulations of the Einstein field equations seem considerably different from the elegant spin-2 solution $\nabla_{A^{\prime}}{ }^A \phi_{A B C D}=0$, which is also highly distinct. The CEFEs' non-linear wave-like formulation was first developed for the vacuum situation in \cite{Pae14}, and it has since been expanded to include the case of trace-free matter in \cite{CarHurKro19}, \cite{FenGas23}.\\
H. Friedrich's conformal Einstein field equations are fundamentally based on equations for the Weyl tensor derived from the Bianchi identities. These formulations represent a distinct departure from purely metric formulations like the ADM, Z4, and generalized harmonic gauge formulations. According to Penrose's perspective on linearized gravity, specifically in the context of vacuum and around flat spacetime, the gravitational field is represented by a spin-2 field that satisfies the following equations:
\begin{equation}\label{eq:Spin2Eq}
  \tilde{\nabla}^{A A^{\prime}} \tilde{\phi}_{A B C D}=0,
\end{equation}
in this context, $\tilde{\phi}_{A B C D}$ denotes a totally symmetric spinor field, which serves as the linear counterpart of the Weyl spinor.
This equation provides an excellent linear model for the Bianchi sector of the conformal Einstein field equations. A similar observation applies to Maxwell's equations in flat spacetime, which can be described in terms of the spin-1 equation
\begin{equation}\label{eq:Spin1Eq}
  \tilde{\nabla}^{A A^{\prime}} \tilde{\phi}_{A B}=0,
\end{equation}
in this equation, $\tilde{\phi}_{A B}$ represents a symmetric spinor known as the Maxwell spinor, which encodes the electromagnetic field \cite{GasPin23}.\\
In \cite{NewPen68}, the NP constants were originally defined within the linear framework for spin-1 and spin-2 fields, as well as within the nonlinear setting for the Einstein field equations formulated in the Newman-Penrose formalism.
While the spin-2 field is often employed in curvature-oriented formulations, wave equations are better suited for metric formulations. In the case of linearized gravity, for instance, the standard formulation, as described in [\cite{Mag07a}, \cite{Wal84a}], typically follows an approach that resembles the hyperbolic reduction of the Einstein field equations in harmonic gauge. In this formulation, the final expression takes the form of a wave equation for the metric components.\\
Scalar fields that satisfy these wave equations can be referred to as spin-0 fields. In this section, we will express the physical wave equation in flat spacetime as a wave equation for an unphysical field propagating in the i0-cylinder background. A method to solve the resulting equation, following a similar approach to that used for spin-1 and spin-2 fields in [\cite{Val07}, \cite{GasKro16d}], has been outlined in [\cite{MinMacKro22}], \cite{GasPin23}.
%%%%%%%%%%%%%%%%%%%%%%%%%%%%%%%%%%%%%%%%%%%%%%%%%%%%%%%%%%%%%%%
\section{Spin-0 fields close to $i^0$ and $\mathscr{I}$}
\label{sec:Spin0FieldsCloseToI0AndI}

Recall that for two conformally related manifolds - which do not necessarily have to be the $i^0$ cylinder and Minkowski spacetime - $(\tilde{M},\tilde{\boldsymbol{g}})$ and $(M,\boldsymbol{g})$, the D'Alembertian operator
transforms under conformal transformations as follows \cite{DuaFenGasHil22}:
\begin{equation}\label{eq:waveConfTr}
	\square \phi-\frac{1}{6} \phi R=\Omega^{-3}\left(\tilde{\square} \tilde{\phi}-\frac{1}{6} \tilde{\phi} \tilde{R}\right),
\end{equation}
where $\tilde{\square}=\tilde{g}^{ab}\tilde{\nabla}_a\tilde{\nabla}_b$ and $\square=g^{ab}\nabla_a\nabla_b$ with $\nabla$ and $\tilde{\nabla}$, ~$R$ and~$\tilde{R}$,
denoting the Levi-Civita connections and Ricci scalars of ~$\boldsymbol{g}$ and~$\tilde{\boldsymbol{g}}$, respectively.  Let $\tilde{\phi}$ be a scalar propagating in
flat spacetime $(\mathbb{R}^4, \tilde{\boldsymbol{\eta}})$ according to:
\begin{equation}\label{eq:wave-eq}
	\tilde{\Box} \tilde{\phi} = 0.
\end{equation}
By applying the conformal transformation formula for the wave equation, given in equation \eqref{eq:waveConfTr}, to the wave equation in \eqref{eq:wave-eq} on the physical Minkowski spacetime $(\tilde{M} , \tilde{\boldsymbol{\eta}})$ and selecting the target conformal extension - the unphysical spacetime - $(M, \boldsymbol{g})$ to be Friedrich's cylinder at spatial infinity, we can obtain the following equation:
\begin{equation}\label{eq:wave-unphysical}
	\square \phi = 0.
\end{equation}
Thus, \eqref{eq:wave-unphysical} is just the wave equation for the unphysical field propagating on the $i^0$ cylinder background.
\pagebreak
\begin{remark}
  In the $i^0$-cylinder representation of Minkowski spacetime expressed in physical coordinates, the conformal factor is given by $\Theta=\tilde{\rho}^{-1}$. As a result, the unphysical (conformal) field $\phi=\Theta^{-1}\tilde{\phi}$ can be directly identified as the \emph{radiation field} $\tilde{\rho}\tilde{\phi}$ due to the factor of $\tilde{\rho}$ present.
\end{remark}

\noindent In $F$-coordinates, equation \eqref{eq:wave-eq} can be explicitly written as:

\begin{equation}\label{eq:UnphysicalWaveExplicit}
  \left(\tau^2-1\right) \partial_\tau^2 \phi-2 \rho \tau \partial_\tau \partial_\rho \phi+\rho^2 \partial_\rho^2 \phi+2 \tau \partial_\tau \phi+\Delta_{\mathbb{S}^2} \phi=0,
\end{equation}
here, $\Delta _{\mathbb{S}^{2}{}}{}$ represents the Laplace operator on $\mathbb{S}^2$. Following the methodology employed in the analysis of spin-1, spin-2, and spin-0 fields in [\cite{ValAli22}, \cite{MinMacKro22}], respectively, we adopt the following Ansatz:
\begin{equation}\label{eq:ansatz}
	\phi = \sum_{p = 0}^{\infty}\sum_{l = 0}^{p}\sum_{m = -l}^{m = l}\frac{1}{p!}a_{p;l,m}(\tau)\rho^{p}Y_{lm}.
\end{equation}
In terms of initial data, this entails examining analytic initial data in the vicinity of $i^0$, which can be expressed in the following form:
\begin{equation}\label{eq:ID_field}
  \phi|_{\mathcal{S}} =
  \sum_{p=0}^{\infty}
  \sum_{\ell=0}^{p}\sum_{m=-\ell}^{m=\ell}\frac{1}{p!}a_{p;\ell,m}(0)Y_{\ell
    m}\rho^p, \qquad \dot{\phi}|_{\mathcal{S}} =
  \sum_{p=0}^{\infty}\sum_{\ell=0}^{p}\sum_{m=-\ell}^{m=\ell}
  \frac{1}{p!}\dot{a}_{p;\ell,m}(0)Y_{\ell
    m}\rho^p,
\end{equation}
here, the over-dot represents a derivative with respect to $\partial_\tau$. A calculation, as detailed in \cite{MinMacKro22}, reveals that solving the wave equation \eqref{eq:UnphysicalWaveExplicit} simplifies to solving the following ordinary differential equation (ODE) for every $p$, $\ell$, and $m$:

\begin{equation}\label{eq:ODE_wave_JacobiPoly}
  (1-\tau^2)\ddot{a}_{p;\ell,m} +
    2\tau(p-1)\dot{a}_{p,\ell,m}+(\ell+p)(\ell-p+1){a}_{p;\ell,m}=0.
\end{equation}

\noindent The solution to this ODE can be expressed as follows

\begin{lemma} (wave equation in the $i^0$-cylinder
  background~ \cite{MinMacKro22})\label{Lemma:Sol_Jacobi_and_Logs}. The solution to equation \eqref{eq:ODE_wave_JacobiPoly} is given
  by:
	\begin{enumerate}
	\item For $p\geq 1$ and $0\leq \ell \leq p-1$
	 \begin{align}\label{eq:Sol_jac_poly}
    a(\tau)_{p;\ell,m} =A_{p,\ell,m}
		  \bigg(\frac{1-\tau}{2}\bigg)^{p}
                  P_{\ell}^{(p,-p)}(\tau) + B_{p,\ell,m}
                  \bigg(\frac{1+\tau}{2}\bigg)^{p}P_{\ell}^{(-p,p)}(\tau)
	 \end{align}
	
	\item For $p\geq 0$ and $\ell=p$:
     \begin{align}\label{eq:Sol_highestharmonic}
      {a}_{p;p,m}(\tau) =
      \bigg(\frac{1-\tau}{2}\bigg)^{p}\bigg(\frac{1+\tau}{2}\bigg)^{p}\Bigg(C_{p,p,m}
      +D_{p,p,m}\int_{0}^{\tau} \frac{ds}{(1-s^2)^{p+1}}\Bigg)
     \end{align}
	where $A_{p,\ell,m}$, $B_{p,\ell,m}$, $C_{p,p,m}$ and
        $D_{p,p,m}$ are constants that can be written in terms of
        $a_{p;\ell,m}(0)$ and $\dot{a}_{p;\ell,m}(0)$ and
        $P_{\gamma}^{\alpha, \beta}(\tau)$ are the Jacobi polynomials.
    \end{enumerate}
\end{lemma}

\pagebreak

One intriguing characteristic shared by the study of evolution equations using the i0-cylinder framework is the emergence of logarithmic terms at null infinity in the solution. This becomes evident when examining the hypergeometric function featured in Equation \eqref{eq:Sol_highestharmonic} for various values of p. For instance, for p = 0 and p = 1, the following expressions arise:
\begin{align}
  {a}_{0;0,0}(\tau) & = C_{000} + \tfrac{1}{2} D_{000} (\log(1 + \tau
  )- \log(1 - \tau ))\\ {a}_{1;1,m}(\tau) & = \tfrac{1}{4} (1 - \tau )
  (1 + \tau ) (C_{11m} + \tfrac{1}{4} D_{11m} ( \log(1 + \tau ) -
  \log(1 - \tau ) + 2\tau(1-\tau^2)))
\end{align}
These logarithmic terms have implications for the linear version of the associated Peeling property, as discussed in \cite{Val07,MinMacKro22}. In the full non-linear case, additional obstacles to the smoothness of null infinity arise, as explored in \cite{Val04}. In the gravitational scenario, conditions can be imposed on the initial data to prevent the emergence of these logarithmic terms in the evolution, as described in \cite{Fri98a}. For the spin-0 case, the corresponding condition can be expressed as follows:

\begin{remark}\label{Remark:logfreeRemark}(Regularity condition~\cite{MinMacKro22}).
  \emph{Lemma \ref{Lemma:Sol_Jacobi_and_Logs}} implies that expanding the integral in \eqref{eq:Sol_highestharmonic} results in logarithmic terms, hence
  $D_{p,p,m} = 0$ is called the regularity condition. The solutions for
  $a(\tau)$ are polynomic in $\tau$, except for $l = p$ where one needs
  to impose the regularity condition to only have polynomic
  solutions. \cite{MinMacKro22}.
\end{remark}
%%%%%%%%%%%%%%%%%%%%%%%%%%%%%%%%%%%%%%%%%%%%%%%%%%%%%%%%%%%%%%%%%%%%%
\chapter{The NP-constants for the spin-0 fields close to $i^0$ \& $\mathscr{I}$}
\label{chapter:NPConstants}


The Newman-Penrose (NP) constants, introduced by Newman and Penrose \cite{NewPen68}, are computed by evaluating 2-surface integrals at null infinity of certain derivatives of the field. These constants, as the name suggests, are conserved quantities, meaning that their values remain constant regardless of the choice of the cut of null infinity on which they are computed. In their original work \cite{NewPen68}, it was demonstrated that for the spin-1 field (Maxwell field) and spin-2 field (linearized Weyl tensor) in Minkowski spacetime, there exists an infinite hierarchy of conserved quantities. However, in the full non-linear theory involving the Weyl tensor, only 10 quantities remain conserved. In both the linear (spin-1 and spin-2) and non-linear (gravitational) cases, the NP constants emerge from a set of asymptotic conservation laws. For the spin-0 field in flat spacetime, refer to \cite{Keh21_a}. One has
\begin{equation}\label{eq:cons_laws}
  {\underline{{L}}}({\tilde{\rho}}^{-2\ell}L(e^{+})^{\ell+1}\phi_{\ell m})=0, \qquad L({\tilde{\rho}}^{-2\ell}\underline{L}(e^{-})^{\ell+1}\phi_{\ell m})=0
\end{equation}
here, $\phi_{\ell m}= \int_{\mathbb{S}^2} \phi  Y_{\ell m}  d\sigma$ represents the integral of $\phi$ multiplied by the spherical harmonics $Y_{\ell m}$ over the surface $\mathbb{S}^2$, where $d\sigma$ denotes the area element on $\mathbb{S}^2$. A derivation of a slightly more generalized version of these identities can be shown in the following section.

%%%%%%%%%%%%%%%%%%%%%%%%%%%%%%%%%%%%%%%%%%%%%%%%%%%%%%%%%%%%%%%%%%%%%
\section{Conservation Laws}
\label{sec:conservationlaws}

The conservation laws \eqref{eq:cons_laws} have been generalized to a Schwarzschild background in \cite{Keh21_a}, and thus the flat space version follows immediately from these results. However, for the sake of clarity and self-contained discussion, we will derive the conservation laws in flat space here.

\begin{proposition}\label{prop:main_commutation}
  Let $\tilde{\phi}$ be a solution to $\tilde{\square}\tilde{\phi}=0.$ Let $\phi = \tilde{\rho}\tilde{\phi}$ then $\phi$ satisfies
  \begin{equation}\label{eq:main_commutation}
     \underline{L} L (\boldsymbol{e}^{+})^n \phi = -\frac{2n}{\tilde{\rho}}L (\boldsymbol{e}^{+})^n
     \phi + \frac{1}{\tilde{\rho}^2}(n(n+1) + \Delta_{\mathbb{S}^2})(\boldsymbol{e}^+)^n \phi.
  \end{equation}
\end{proposition}
\pagebreak
\begin{proof}
  This identity can be proven through an induction process. In terms of $\underline{L}$, $L$, and $\phi$, the equation $\tilde{\square}\tilde{\phi}=0$ can be expressed as follows:

\begin{align}\label{eq:uLLphiToSource}
  \underline{L} L \phi = \frac{1}{\tilde{\rho}^2}(\Delta_{\mathbb{S}^2} \phi)
\end{align}
Equation \eqref{eq:uLLphiToSource} corresponds to the case $n=0$ of expression \eqref{eq:main_commutation} and serves as the basis for the induction. Assuming that expression \eqref{eq:main_commutation} holds true for $n=N$, let's proceed with the induction step by calculating $\underline{L} L (\boldsymbol{e}^+)^{n+1} \phi$ as follows:

% change the vertical spacing between lines of the align environment
\setlength{\jot}{10pt}


\begin{align}\label{eq:inductionstep_commutation}
  & \underline{L} L\left(e^{+}\right)^{N+1} \phi=\underline{L} L\left\{\left(e^{+}\right)\left[\left(e^{+}\right)^{N} \phi\right]\right\}=L\left[\underline{L}\left[\left(p^{2} L\right)\left(e^{+}\right)^{N} \phi\right]\right]= \nonumber\\ 
  & =L\left[\underline{L}\left(\rho^{2} L\left(e^{+}\right)^{N} \phi\right)\right]=L\left[\tilde{\rho}^{2} \underline{L} L\left(e^{+}\right)^{N} \phi-2 \tilde{\rho} L\left(e^{+}\right)^{N} \phi\right]= \nonumber \\
  & =L\left[-2 N \tilde{p} L\left(e^{+}\right)^{N} \phi+(N(N+1)+\Delta)\left(e^{+}\right)^{N} \phi-2 \tilde{p} L\left(e^{+}\right)^{N} \phi\right]= \nonumber \\
  & =-2 N \tilde{p} L L\left(e^{+}\right)^{N} \phi-2 N L\left(e^{+}\right)^{N} \phi+L\left[\left(N(N+1)\right)\left(e^{+}\right)^{N} \phi\right]+L\left(\Delta\left(e^{+}\right)^{N} \phi\right) -2 \tilde{p} L L\left(e^{+}\right)^{N} \phi-2 L\left(e^{+}\right)^{N} \phi= \nonumber \\
  & =-2 N\tilde{\rho} L L\left(e^{+}\right)^{N} \phi-2 N L\left(e^{+}\right)^{N} \phi+(N(N+1)) L\left(e^{+}\right)^{N} \phi + \nonumber \\
  & + \left(e^{+}\right)^{N} \phi L(N(N+1))+\Delta L\left(e^{+}\right)^{N} \phi-2 \tilde{\rho} L L\left(e^{+}\right)^{N} \phi-2 L\left(e^{+}\right)^{N} \phi= \nonumber \\
  & =-2 N\tilde{\rho} L L\left(e^{+}\right)^{N} \phi-2 N L\left(e^{+}\right)^{N} \phi+N^{2} L\left(e^{+}\right)^{N} \phi+N L\left(e^{+}\right)^{N} \phi+\Delta L\left(e^{+}\right)^{N} \phi -2 \tilde{p} L L\left(e^{+}\right)^{N} \phi-2 L\left(e^{+}\right)^{N} \phi = \nonumber \\
  & =-2 N\tilde{\rho} L L\left(e^{+}\right)^{N} \phi-N L\left(e^{+}\right)^{N} \phi+N^{2} L\left(e^{+}\right)^{N} \phi +\Delta L\left(e^{+}\right)^{N} \phi-2 \tilde{\rho} L L\left(e^{+}\right)^{N} \phi-2 L\left(e^{+}\right)^{N} \phi.
\end{align}
The RHS of \eqref{eq:main_commutation} with $n = N+1$ can be broken down into individual components and computed individually,
\begin{align}\label{eq:RHS1}
  -\frac{2(N+1)}{\tilde{\rho}}L\left(e^{+}\right)^{N+1} \phi =-2 N \tilde{\rho} L L\left(e^{+}\right)^{N} \phi-4 N L\left(e^{+}\right)^{N} \phi-2\tilde{\rho}  L L\left(e^{+}\right)^{N} \phi-4 L\left(e^{+}\right)^{N} \phi
\end{align}
\begin{align}\label{eq:RHS2}
  \frac{1}{\tilde{\rho}^2}((N+1)(N+2) + \Delta_{\mathbb{S}^2})(\boldsymbol{e}^+)^{N+1} \phi = N^{2} L\left(e^{+}\right)^{N} \phi+3 N L\left(e^{+}\right)^{N} \phi+2 L\left(e^{+}\right)^{N} \phi+\Delta L\left(e^{+}\right)^{N} \phi
\end{align}
Adding \eqref{eq:RHS1} and \eqref{eq:RHS2} together, we obtain
\begin{align}\label{eq:RHS1plusRHS2}
  -2 N \tilde{\rho} L L\left(e^{+}\right)^{N} \phi+N^{2} L\left(e^{+}\right)^{N} \phi-N L\left(e^{+}\right)^{N} \phi-2 L\left(e^{+}\right)^{N} \phi - 2 \tilde{\rho} L L\left(e^{+}\right)^{N} \phi+\Delta L\left(e^{+}\right)^{N} \phi
\end{align}
Noticing that \eqref{eq:RHS1plusRHS2} is the same as expression \eqref{eq:main_commutation} with $n = N + 1$ finishes the proof.
\end{proof}
Furthermore, it is important to note that since this calculation is performed in flat spacetime, we have made trivial commutations involving $\underline{L}$, $L$, and $\Delta_{\mathbb{S}^2}$.
Using Proposition \ref{prop:main_commutation} for $\ell=n$ and rearranging gives the following:
\pagebreak
\begin{corollary}\label{coro:main_commutation}
  Let $\tilde{\phi}$ be a solution to
  $\tilde{\square}\tilde{\phi}=0$ and let $\phi =
  \tilde{\rho}\tilde{\phi}$. Then
  \begin{align}\label{eq:coromain_commutation}
  \underline{L} (\tilde{\rho}^{-2\ell} L (\boldsymbol{e}^{+})^{\ell}\phi_{\ell m}) = 0
  \end{align}
  where $\phi_{\ell m}= \int_{\mathbb{S}^2} \phi \; Y_{\ell m} \;
  d\sigma$ with $d\sigma$ denoting the area element in
  $\mathbb{S}^2$.
\end{corollary}
\begin{proof}
  Using \eqref{eq:main_commutation} with $n = N$ and substituting $\phi=\sum_{\ell, m} \phi_{\ell M}(\tau, \rho) Y_{l m}$ into the same expression leads to,
  \begin{equation}\label{eq:coromain_commutation_proof}
    \underline{L} L\left(e^{+}\right)^{N}\left(\sum_{l, m} \phi_{l n} Y_{l m}\right)=-\frac{2 N}{\tilde{p}} L\left(e^{+}\right)^{N}\left(\sum_{l, m} \phi_{l m} Y_{l m}\right)+\frac{1}{\tilde{\rho}^{2}}(N(N+1)+\Delta)\left(e^{+}\right)^{N}\left(\sum_{l, M} \phi_{l n} Y_{l m}\right)
  \end{equation}
  Applying the \textit{Liebniz} rule when the $L$ operator acts on $\phi_{\ell m}$ and $Y_{\ell m}$, gives the following:
  \begin{equation}\label{eq:liebnizrule}
    L\left(\phi_{l m} Y_{l m}\right):=\phi_{l m} L Y_{l m}+Y_{l m} L \phi_{l m}
  \end{equation}
  The operator $L$ acting on $Y_{\ell m}$ gives $0$ since $Y_{\ell m} = Y_{\ell m}(\theta, \varphi)$ depends only on angular derivatives. Therefore, \eqref{eq:liebnizrule} becomes
  \begin{equation}\label{eq:liebnizrule2}
    L\left(\phi_{l m} Y_{l m}\right)=Y_{l m} L (\phi_{l m})
  \end{equation}
  Using \eqref{eq:liebnizrule2} and $N = \ell$ we can rewrite \eqref{eq:coromain_commutation_proof} as,
  \begin{equation}\label{eq:coromain_commutation_proof2}
    \sum_{l, m} Y_{l m}\left[\underline{L} L\left(e^{+}\right)^{l} \phi_{l m}+\frac{2l}{\tilde{\rho}} L\left(e^{+}\right)^{l} \phi_{l m}\right]=\frac{1}{\tilde{\rho}^{2}}\left[(l(l+1))\left(e^{+}\right)^{l}\left(\sum_{l, m} \phi_{l m} Y_{l m}\right) +\left(\sum_{l_{1} m} \phi_{l m} \Delta Y_{l m}\right)\left(e^{+}\right)^{l}\right]
  \end{equation}
  Using the fac that $\sum_{l_{1} m} \Delta \phi_{l m} Y_{l m}=0$ and $\sum_{l, m} \phi_{l m} Y_{l m}=-l(l+1) \sum_{l, m} \phi_{l m}$ \eqref{eq:coromain_commutation_proof2} becomes,
  \begin{align}\label{eq:coromain_commutation_proof3}
    & \sum_{l, m} Y_{l m}\left(\underline{L} L\left(e^{+}\right)^{l} \phi_{l m}+\frac{2 l}{\tilde{p}} L\left(e^{+}\right)^{l} \phi_{l m}\right)=0 \Leftrightarrow \nonumber \\
    & \Leftrightarrow \underline{L} L\left(e^{+}\right)^{l} \phi_{l m}+\frac{2 l}{\tilde{\rho}} L\left(e^{+}\right)^{l} \phi_{l m}=0 \quad \Leftrightarrow \nonumber \\
    & \Leftrightarrow \underline{L}\left(\tilde{\rho}^{-2 l} L\left(e^{+}\right)^{l} \phi_{l m}\right)=0
  \end{align}
  We can see that \eqref{eq:coromain_commutation_proof3} is the same as \eqref{eq:coromain_commutation}, which completes the proof.   
\end{proof}
Then, with the necessary adjustments, the time-reversed versions of \ref{prop:main_commutation} and \ref{coro:main_commutation} can be expressed as:
\begin{proposition}\label{prop:main_commutation_minus}
  Let $\tilde{\phi}$ be a solution to
  $\tilde{\square}\tilde{\phi}=0$.  Let $\phi =
  \tilde{\rho}\tilde{\phi}$ then $\phi$ satisfies
 \begin{align}\label{eq:main_commutation_minus}
   L \underline{L} (\boldsymbol{\underline{e}}^{-})^n \phi = -\frac{2n}{\tilde{\rho}}\underline{L}
   (\boldsymbol{\underline{e}}^{-})^n \phi + \frac{1}{\tilde{\rho}^2}(n(n+1) +
   \Delta_{\mathbb{S}^2})(\boldsymbol{\underline{e}}^{-})^n \phi.
 \end{align}
\end{proposition}
\begin{corollary}\label{coro:main_commutation_minus}
  Let $\tilde{\phi}$ be a solution to
  $\tilde{\square}\tilde{\phi}=0$ and let $\phi =
  \tilde{\rho}\tilde{\phi}$. Then
  \begin{align}\label{eq:coromain_commutation_minus}
  L (\tilde{\rho}^{-2\ell} \underline{L} (\boldsymbol{\underline{e}}^{-})^{\ell}\phi_{\ell m}) = 0
  \end{align}
  where $\phi_{\ell m}= \int_{\mathbb{S}^2} \phi \; Y_{\ell m} \;
  d\sigma$ with $d\sigma$ denoting the area element in $\mathbb{S}^2$.
\end{corollary}
%%%%%%%%%%%%%%%%%%%%%%%%%%%%%%%%%%%%%%%%%%%%%%%%%%%%%%%%%%%%%%%%%%%%%%%%%%%%
\section{Definition of NP constants}
\label{sec:NP_constants}

Equations \eqref{eq:cons_laws} provide an infinite hierarchy of exact conservation laws. Additionally, one can introduce the $f(\tilde{\rho})$-modified NP constants, as described in \cite{GajKehLeo22}, in the following manner: 
\begin{align}\label{eq:DefModifiedNP}
  {}^{f}\mathcal{N}^{+}_{\ell,m}:= f(\tilde{\rho})L (\boldsymbol{e}^{+})^{\ell}\phi_{\ell m} \Big|_{\mathcal{C}^{+}}, \\ 
  {}^{f}\mathcal{N}^{-}_{\ell,m}:= f(\tilde{\rho})\underline{L} (\boldsymbol{\underline{e}}^{-})^{\ell}\phi_{\ell m}\Big|_{\mathcal{C}^{-}},
\end{align}
here, $\mathcal{C^{\pm}} \approx \mathbb{S}^2$ represents a cut of $\mathscr{I}^{\pm}$. In the specific case of $f(\tilde{\rho})=\tilde{\rho}^2$, these quantities are referred to as the "classical NP-constants" and are denoted as $\mathcal{N}^{\pm}_{\ell,m}$. They can be succinctly expressed as follows:
\begin{align}\label{eq:classicalNP}
  \mathcal{N}^{+}_{\ell,m}:= (\boldsymbol{e}^{+})^{\ell+1}\phi_{\ell m}\Big|_{\mathcal{C}^{+}},\\ 
  \mathcal{N}^{-}_{\ell,m}:= (\boldsymbol{\underline{e}}^{-})^{\ell+1}\phi_{\ell m} \Big|_{\mathcal{C}^{-}}.
\end{align}
In the notation for ${}^{f}\mathcal{N}^{+}_{\ell,m}$ presented below, it will be implicitly assumed that $m$ takes values from $-\ell$ to $\ell$.
%%%%%%%%%%%%%%%%%%%%%%%%%%%%%%%%%%%%%%%%%%%%%%%%%%%%%%%%%%%%%%%%%%%%%%%%%%%%
\section{The classical NP constants}
\label{sec:classicalNP}

In this section, we focus on computing the classical NP constants, which are determined by the initial data expressed through the constant parameters discussed in Lemma \ref{Lemma:Sol_Jacobi_and_Logs}. To aid our comprehension, we begin by calculating a few initial NP constants before proceeding to the general case. Specifically, we delve into computing the classical NP constants at $\mathscr{I}^{+}$ for $\ell=0$ and $\ell=1$. This analysis is facilitated by expression \eqref{eq:ansatz}, which yields the following insights:
\begin{align}\label{eq:exp_phi_lm}
  \phi_{\ell m}= \sum_{p=\ell}^{\infty}\frac{1}{p!}a_{p;\ell,m}(\tau)\rho^{p}.
\end{align}
Therefore, when considering $\ell=0$, the computation of $\boldsymbol{e}^{+}(\phi_{00})$ is sufficient. By utilizing Proposition \ref{Prop:NPtoFgauge} and equation \eqref{eq:Fframe}, we obtain the following:
\begin{align}\label{eq:bmeplus1philmraw}
   & \boldsymbol{e}^{+}(\phi_{\ell m})= 4(\Lambda_{+})^{2}\sum_{p=0}^{\infty}\frac{1}{p!}\boldsymbol{e}(a_{p;\ell,m}(\tau)\rho^{p}) = \Lambda_{+}^{2} \sum_{p=0}^{\infty} \left[(1+\tau) \partial_{\tau}-p \partial_{\rho}\right]\left(a_{p ; \ell, m}(\tau) \rho^{p}\right)= \nonumber \\
   & =\Lambda_{+}^{2} \sum_{p=0}^{\infty} \left[(1+\tau) \dot{a}_{p_; l, m} \rho^{p}-\rho a_{p ; l, m} \cdot p \rho^{p-1}\right] = 4 \rho^{-1}(1+\tau)^{-2}\sum_{p=0}^{\infty} \frac{1}{p!}\rho^p((1+\tau)\dot{a}_{p;\ell,m}-p a_{p;\ell,m}).
\end{align}
With
\begin{align}\label{eq:defQ0}
  Q^{0}_{p;\ell,m}(\tau):=(1+\tau)\dot{a}_{p;\ell,m}-p a_{p;\ell,m},
\end{align}
With this definition in place, we can express $\boldsymbol{e}^{+}(\phi_{\ell m})$ as follows:
\begin{align}\label{eq:bmeplus1philm}
  \boldsymbol{e}^{+}(\phi_{\ell m}) = 4 (\Lambda_{+})^{2}\sum_{p=0}^{\infty} \frac{1}{p!}\rho^{p}Q^{0}_{p,\ell,m}(\tau).
\end{align}
To compute the $\ell=0$ NP constant at $\mathscr{I}^{+}$, it is necessary to evaluate $\boldsymbol{e}^{+}(\phi_{00})$ at a specific cut $\mathcal{C}^{+}$ of $\mathscr{I}^{+}$. By utilizing equation \eqref{eq:bmeplus1philm} and referring to Lemma \ref{Lemma:Sol_Jacobi_and_Logs}, we obtain the following expression:
\begin{align}
  \mathcal{N}^{+}_{0,0}= \lim_{\substack{\rho \to \rho_{\star} \\ \tau \to 1}}  \boldsymbol{e}^{+}(\phi_{00}) = \sum_{p=0}^{\infty} \frac{1}{p!}\rho^{p-1}_{\star}Q^{0}_{p,0,0}|_{\mathscr{I}^{+}}.
\end{align}
here, $\rho_{\star}$ represents a constant that parametrizes the cut $\mathcal{C}^{+}$, and $Q^{0}_{p,\ell,m}|_{\mathscr{I}^{+}}$ denotes the value of $Q^{0}_{p,\ell,m}$ at $\mathscr{I}^{+}$, specifically at $\rho=1$. Notably, when $\rho_{\star}=0$, it corresponds to selecting $\mathcal{C}^{+}$ as $\mathcal{I}^{+}$.
A direct calculation using Lemma \ref{Lemma:Sol_Jacobi_and_Logs} shows the following:
\begin{align}\label{eq:Q0lm}
  & Q_{p;l,m}^{0}(\tau)=(1+\tau) \dot{a}_{p ; l, m}(\tau)-p_{p ; l, m}(\tau) \Rightarrow Q_{0;0,0}^{0}(\tau)=2 \dot{a}_{0;0,0}(\tau)-0 \cdot a_{0;0,0} =2 \dot{a}_{0;0,0}(\tau)=\frac{D_{000}}{1-\tau}, \\
  & Q_{1;0,0}^{0}(\tau)=2 \dot{a}_{1;0,0}(1)-1 \cdot a_{1;0,0}(1)=-A_{100}.
\end{align}
Therefore, it is important to note that if the regularity condition of Remark \ref{Remark:logfreeRemark} is not satisfied, the classical NP constants will not be well-defined, regardless of the chosen cut for evaluation. Thus, in order to compute the classical $\ell=0$ NP constant, the regularity condition must be imposed. However, once the regularity condition is satisfied, the value of the classical $\ell=0$ NP constant becomes independent of the specific cut chosen for evaluation. This can be seen through the following analysis:
\begin{remark}\label{rem:l0}
  A direct calculation
using Lemma \ref{Lemma:Sol_Jacobi_and_Logs}
gives
 \begin{subequations}\label{eq:rem:l0}
 \begin{align}
   Q^{0}_{0;0,0}&=D_{000}(1-\tau)^{-1},\label{rem:l0:eq1} \\
   Q^{0}_{p;0,0}&=-2^{1-p}pA_{p,0,m}(1-\tau)^{p-1} \qquad \text{ for}\qquad p\neq 0.
   \label{rem:l0:eq2}
 \end{align}
\end{subequations}
\end{remark}
Thus, based on the insights provided by Remark \ref{rem:l0} and assuming the regularity condition is fulfilled, we can conclude that:
\begin{align}
  &\mathcal{N}^{+}_{0,0}= \sum_{p=0}^{\infty} \frac{1}{p!}\rho^{p-1}_{\star}Q^{0}_{p;0,0}|_{\mathscr{I}^{+}} = \sum_{p=0}^{\infty} \frac{1}{p !}(1+\tau)^{-2} \rho^{p-1} Q_{p; 0,0}^{0}(\tau) = \sum_{p=0}^{\infty} \frac{1}{p !} 2^{-2} \rho^{p-1} Q_{p; 0,0}^{0}(1) \nonumber \\
  & \left[\frac{2^{-2}}{0 !} \rho^{-1} A_{0;0,0}\cdot(1)+\frac{2^{-2}}{\rho !} \rho^{0} A_{1;0,0} \cdot(1)+\sum_{p=2}^{\infty} \frac{2^{-2}}{p !} \rho^{p-1} A_{p;0,0}\cdot(\tau)\right] = -A_{100}.
\end{align}

It is expected that the regularity condition must be imposed to ensure well-defined classical NP constants, as a similar observation holds for the spin-1 and spin-2 cases in \cite{GasVal20} (see also \cite{Val98, Val99a}). Additionally, Remark \ref{rem:l0} demonstrates that only the term with $p=1$ contributes to the calculation. This aligns with the "constancy" of the NP constants, indicating that their value does not depend on the chosen cut \cite{NewPen68}. In other words, if we assume that the NP constants are well-defined, the specific form of the expressions in Remark \ref{rem:l0} is not necessary to determine the contribution of the initial data parameters from Lemma \ref{Lemma:Sol_Jacobi_and_Logs} to $\mathcal{N}^{+}_{0,0}$. As long as only the parameter $A_{p;\ell,m}$ (and not $B_{p;\ell,m}$) appears in Equation \ref{rem:l0:eq2}, the value of $\mathcal{N}^{+}_{0,0}$ can be determined.

